\chapter{Professional Growth}

\section{Technical Aspects}

During my internship, I worked on projects using tools and technologies that were common in the industry but about which I knew much less before.
I'll talk about the tools and technologies I've learned more about in this section.

\subsection{Typescript}

TypeScript is a programming language developed and maintained by Microsoft.
It is a strict syntactical superset of JavaScript and adds optional static typing to the language.
It is designed for the development of large applications and transpiles to JavaScript.\\

I used JavaScript in the projects I developed before my internship.
But it is harder to debug Javascript code during development.
So I had learn and use TypeScript to write my code during my internship.

\subsection{ReactJS}

React is a free and open-source front-end JavaScript library for building user interfaces based on UI components.
I used ReactJS before my internship too, but I could do it in a professional way following the best practices.
During my internship I used to read my seniors' code and understand how they worked.
That gave me good insight into how ReactJS works and how to write better code using ReactJS.

\subsection{Redux}

Redux is an open-source JavaScript library for managing and centralizing application state.
It is most commonly used with libraries such as React or Angular for building user interfaces.
Although it is a little complex to use, it is a great tool for managing state in a complex application.
My seniors guided me through how does it work, I used it with ease.

\subsection{Git}

Git is software for tracking changes in any set of files, usually used for coordinating work among programmers collaboratively developing source code during software development.
I used Git before my internship for my personal projects.
But during my internship I learned how to use Git in a professional way.
I learned about several Git commands that I didn't know of before and sometimes use tools like GMaster, GitLense in VS Code etc to make my work easier.

\subsection{Visual Studio Code}

Visual Studio Code, also commonly referred to as VS Code, is a open-source IDE.
This has features include support for debugging, syntax highlighting, intelligent code completion, snippets, code refactoring, and embedded Git.
I wrote most of my code in Visual Studio Code during my internship period.
This tool have made writing code easier and more efficient using the existing features and extension.

\subsection{Others}

I have learned and used so many new technologies during my internship.
Sometimes I had to do research about new technologies like Docker, Kubernetes, Micro-Frontends, Single-SPA, Cypress etc.
I did research about how to increase performance of a existing front-end.
I also learned how to use Docker to run my projects on a Docker container.
I learned about unit and integration testing using Jest and Cypress.


\section{Non Technical Aspects}

The non-technical skills can be acquired by interacting with the others.
Some of the non-technical or soft skills I have improved are discussed in this section.

\subsection{Code Quality}

We were always encouraged to write elegant and simple code in Kaz.
Kaz adheres to a very high standard of pure software engineering, and their products are of very high quality.
My seniors always reviewed my code and offered suggestions for improvements.
And I made an effort to keep up the quality of my work.

\subsection{Professionalism}

Professionalism is how someone acts, behaves, and thinks in a work or business setting.
It is a way for someone to demonstrate respect for the policies and standards of the company they work for.
The environment for universities and the environment for businesses are very dissimilar.
People in Kaz are incredibly intelligent and always knew what they were doing and talking about.
These kinds of attitudes consistently convince me to be mindful of what I am doing and motivate me to learn more.

\subsection{Team Work}

In the university, we often worked in small groups of two or three people to complete projects.
However, we had to work in a much bigger team during my internship.
I gained experience working with a large group of people, communicating with them, and both giving and receiving help.

\subsection{Punctuality}

In the professional world, punctuality is a crucial issue.
We had to show up for meetings at Kaz Software by a certain time in order to avoid late joining fees.
I consistently arrived on time for meetings.

\subsection{Communication}

One of the fundamental abilities needed to function well in a professional setting is communication skill.
I picked up clear and concise communication skills at Kaz Software, but more importantly, I learned how to ask questions and receive help and support.

\subsection{Planning}

The secret to performing a task more effectively is planning.
I occasionally had to estimate the time needed to complete a project.
I was able to plan my work more effectively as a result.

\subsection{Knowledge Sharing}

I was able to ask anyone any question at any time during my internship as well full time job in Kaz Software.
Everyone always assisted me by giving us the right information and guidance and pushed me to learn more.
Additionally, we used to have conversations with team members to share information about what we were learning.

