\chapter{Professional Growth}

\section{Technical Aspects}

During my internship period I did work in two different projects.
Both of the projects used industry standard tools and technologies, that I had a much less knowledge before.
In this section I will discuss the tools and technologies I have improved my knowledge on.

\subsection{Typescript}

TypeScript is a programming language developed and maintained by Microsoft.
It is a strict syntactical superset of JavaScript and adds optional static typing to the language.
It is designed for the development of large applications and transpiles to JavaScript.\\

I used JavaScript in the projects I developed before my internship.
But it is harder to debug Javascript code during development.
So I had learn and use TypeScript to write my code during my internship.

\subsection{ReactJS}

React is a free and open-source front-end JavaScript library for building user interfaces based on UI components.
I used ReactJS before my internship too, but I could do it in a professional way following the best practices.
During my internship I used to read my seniors' code and understand how they worked.
That gave me good insight into how ReactJS works and how to write better code using ReactJS.

\subsection{Redux}

Redux is an open-source JavaScript library for managing and centralizing application state.
It is most commonly used with libraries such as React or Angular for building user interfaces.
Although it is a little complex to use, it is a great tool for managing state in a complex application.
My seniors guided me through how does it work, I used it with ease.

\subsection{Git}

Git is software for tracking changes in any set of files, usually used for coordinating work among programmers collaboratively developing source code during software development.
I used Git before my internship for my personal projects.
But during my internship I learned how to use Git in a professional way.
I learned about several Git commands that I didn't know of before and sometimes use tools like GMaster, GitLense in VS Code etc to make my work easier.

\subsection{Visual Studio Code}

Visual Studio Code, also commonly referred to as VS Code, is a open-source IDE.
This has features include support for debugging, syntax highlighting, intelligent code completion, snippets, code refactoring, and embedded Git.
I wrote most of my code in Visual Studio Code during my internship period.
This tool have made writing code easier and more efficient using the existing features and extension.

\subsection{Others}

I have learned and used so many new technologies during my internship.
Sometimes I had to do research about new technologies like Docker, Kubernetes, Micro-Frontends, Single-SPA, Cypress etc.
I did research about how to increase performance of a existing front-end.
I also learned how to use Docker to run my projects on a Docker container.
I learned about unit and integration testing using Jest and Cypress.


\section{Non Technical Aspects}

The non-technical skills can be acquired by interacting with the others.
Some of the non-technical or soft skills I have improved are discussed in this section.

\subsection{Code Quality}

In Kaz, they always encouraged us to write beautiful and simple code.
Kaz follows a great standard of pure software engineering and their product quality is very high.
My seniors always reviewed my code, and suggested if it can be better.
And I tried to maintain the standard of work from my side.

\subsection{Professionalism}

Professionalism is the conduct, behavior, and attitude of someone in a work or business environment.
It is the way that a person shows respect for the rules and expectations of the organization he or she works for.
The environment for universities is very different from the environment for companies.
In Kaz, people are quite smart and they always knew what they were doing and talking about.
These type of attitudes always persuade me to be aware of what I am doing and encouraged me to learn  more about things.

\subsection{Team Work}

In university, we used to build projects in small groups of either two or three people.
But during my internship, we had to work in a much larger team.
I learned how to work in a big team, communicate with them, help others and get help from them.

\subsection{Punctuality}

Punctuality is a important issue in professional life.
In Kaz Software, we had to attend meetings within a certain time to prevent late joining fees.
I was always on time to attend the meetings.

\subsection{Communication}

Communication skill is one of the basic skills that is required to perform effectively in a professional environment.
In Kaz Software, I learned how to approach people with a clear and concise manner, more importantly how to ask anything or get help.

\subsection{Planning}

Planning is the key to do a job more efficiently.
In certain times I had to give a time estimation of a project.
That helped me to plan my work more efficiently.

\subsection{Knowledge Sharing}

In Kaz Software, I could ask anything at any point to anyone.
Everyone always helped us providing proper content and direction and encouraged me to learn more.
And we used to had chat with others inside the team sharing knowledge about something we learned.

