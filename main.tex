
% !TEX program = xelatex
\documentclass[11pt, a4paper]{report}

% Size of the margins
\usepackage[a4paper,top=3cm,bottom=3cm,left=4cm,right=4cm]{geometry} 
% Font size
% \usepackage[fontsize=1pt]{scrextend}
\usepackage{fontspec}
\setmainfont{Times New Roman}
% For vscode errors
\usepackage{lmodern}
% Language of the text
\usepackage[english]{babel}
% Language for the bibliography
\usepackage[fixlanguage]{babelbib}
% Text encoding
% \usepackage[utf8]{inputenc} 
% Text encoding
% \usepackage[T1]{fontenc}
% Allows you to generate dummy text. It was useful to me
% to understand what the setting of the
% text on the page before I wrote a certain paragraph
\usepackage{lipsum}
% To rotate images
\usepackage{rotating}
% To change the page header
\usepackage{fancyhdr}               

% Mathematical libraries
\usepackage{amssymb}
\usepackage{amsmath}
\usepackage{amsthm}         

% Use of images
\usepackage{graphicx}
% Use of colors
\usepackage[dvipsnames]{xcolor}         
% Using listings for code
\usepackage{listings}          
% To insert hyperlinks between the various elements of the text
\usepackage{hyperref}     
% Different types of underlines
\usepackage[normalem]{ulem}
% remove indent
\setlength\parindent{0pt}


% -----------------------------------------------------------------

% Change the header style
\pagestyle{fancy}
\fancyhf{}
\fancyfoot[C]{\thepage}
\renewcommand{\headrulewidth}{0pt}

% Removes the page number at the beginning of chapters
% \fancypagestyle{plain}{
%   \fancyfoot{}
%   \fancyhead{}
% }

% Code style
\lstdefinestyle{codeStyle}{
    % Comments color
    commentstyle=\color{teal},
    % Color of the keywords
    keywordstyle=\color{Magenta},
    % Line number style
    numberstyle=\tiny\color{gray},
    % Color of the strings
    stringstyle=\color{violet},
    % Text size and style
    basicstyle=\ttfamily\footnotesize,
    % newline only at whitespaces
    breakatwhitespace=false,     
    % newline yes / no
    breaklines=true,                 
    % Position of the caption, top / bottom
    captionpos=b,                    
    % Preserves spaces in code, useful for indentation
    keepspaces=true,
    % Where to display line numbers
    numbers=left,
    % Distance between line numbers
    numbersep=5pt,
    % Show whitespace or not
    showspaces=false,
    % Show whitespace in strings
    showstringspaces=false,
    % Show tabs
    showtabs=false,
    % Size of tabs
    tabsize=2
} \lstset {style = codeStyle}

% Code style for larger size, where I needed smaller text (for example if you want to insert code that has very long lines). To use this style rather than the previous one, use

% \lstset {style=longBlock}
%% enter code ...
% \lstset {style=codeStyle}

% The second command returns to the previous style
\lstdefinestyle{longBlock}{
    commentstyle=\color{teal},
    keywordstyle=\color{Magenta},
    numberstyle=\tiny\color{gray},
    stringstyle=\color{violet},
    basicstyle=\ttfamily\scriptsize,
    breakatwhitespace=false,         
    breaklines=true,                 
    captionpos=b,                    
    keepspaces=true,                 
    numbers=left,                    
    numbersep=5pt,                  
    showspaces=false,                
    showstringspaces=false,
    showtabs=false,                  
    tabsize=2
} \lstset{style=codeStyle}

% By removing the comment on the following command, the sources present in Bibliography.bib but not directly cited with the \ cite command are also inserted in the bibliography
% \ nocite {*}

% Margins before and after blocks of code, to have more distance
\lstset{aboveskip=20pt, belowskip=20pt}

% Change the style of the references, with the text in cyano
\hypersetup{
    colorlinks,
    linkcolor=black,
    citecolor=black
}

% Added definitions, theorems, line and listing
\newtheorem{definition}{Definizione}[section]
\newtheorem{theorem}{Teorema}[section]
\providecommand*\definitionautorefname{Definizione}
\providecommand*\theoremautorefname{Teorema}
\providecommand*{\listingautorefname}{Listing}
\providecommand*\lstnumberautorefname{Linea}

\raggedbottom

%\newcommand{\cgs}[1]{{\textcolor{brown}[\textcolor{red}{\bf{GS: }}{ \textcolor{brown}{#1]}}}}             
%\newcommand{\cmc}[1]{{\textcolor{blue}[\textcolor{magenta}{\bf{MC: }}{ \textcolor{blue}{#1]}}}}



% -----------------------------------------------------------------
\begin{document}

\begin{titlepage}
\begin{figure}[!htb]
    \centering
    \includegraphics[keepaspectratio=true,scale=2]{images/Frontpage/sustlogo.png}
\end{figure}

\begin{center}
    \LARGE{SHAHJALAL UNIVERSITY OF SCIENCE AND TECHNOLOGY}
    \vspace{5mm}
    \\ \LARGE{Institute of Information and Communication Technology}
    % \vspace{5mm}
    % \\ \LARGE{SWE 420}
\end{center}

\vspace{15mm}
\begin{center}
    {\LARGE{\bf SWE 420\\ \vspace{5mm} Report of Internship }}
    
    % Se il titolo è abbastanza corto da stare su una riga, si può usare
    
    % {\LARGE{\bf Un fantastico titolo per la mia tesi!}}
\end{center}
\vspace{30mm}

\begin{minipage}[t]{0.6\textwidth}
	{\large{Submitted By:}{\normalsize\vspace{3mm}
	\bf\\ \large{Md. Shakirul Hasan Khan Mobin} \normalsize\vspace{2mm} \\ Department of Software Engineering \\ 2017831034}}
\end{minipage}
\hfill
\begin{minipage}[t]{0.35\textwidth}\raggedleft
	{\large{Performed At:}{\normalsize\vspace{3mm} \bf\\ \large{Kaz Software}\normalsize\vspace{2mm}\\ 35/5, Shantinagar \\ Dhaka-1217}}
\end{minipage}

\vspace{30mm}
\hrulefill
\\\centering{Date of Submission: 16 July, 2022}

\end{titlepage}
\chapter*{Letter of Transmittal}

July 16, 2022\\
The Director\\
Institute of Information and Communication Technology\\
Shahjalal University of Science and Technology\\
Sylhet, Bangladesh\\\\
Dear Sir,\\
I am very much delighted to submit a report as a part of my internship program.
I am so thankful to the Institute of Information and Communication Technology, SUST for giving me the opportunity to connect my academic knowledge with the latest software development trends in a renowned software industry and to gather great industry experience.\\
I have been working as an intern at Kaz Software as a part of our course SWE 420, starting from September 1, 2021, to February 28, 2022.
This report is based on my learning and experience during my internship time period. And this report covers the technical skills I developed during my internship program as well as my project participation, skill acquisition, and other improvements and issues.\\
I believe this report will be able to summarize the overall outcome of my internship course.
I will be grateful if you accept my report, and your consideration will be highly appreciated.\\\\
Sincerely yours,\\
\begin{figure}[h]
    \includegraphics[width= 0.25\textwidth]{images/LetterOfTransmittal/mySignCropped.jpg}  
    \label{fig:mySign}
\end{figure}\\
Shakirul Hasan Khan\\
Registration No: 2017831034\\
Department of Software Engineering\\
IICT, SUST
\chapter*{Letter of Endorsement\\{\Large\normalfont To whom it may concern}}

% (To whom it may concern)\\\\
\textbf{Subject:} Approval of the report
\vspace{40pt}

This is to certify that all the information mentioned in the report is accurate and not confidential to the company.
Shakirul Hasan Khan had direct involvement in the projects mentioned in the report.

I personally went through and examined every page of the report, and I was unable to locate any information that contravened the organization's privacy policies.

I therefore declare the report to be true and give Shakirul Hasan Khan my best wishes for a prosperous future.

\vspace{75pt}
\setlength{\tabcolsep}{15pt}
\begin{center}
    \begin{tabular}{ccc}
    \cline{1-1} \cline{3-3}
    \begin{tabular}[c]{@{}c@{}}\\Shawal Siddique Shaon\\Chief Technology Officer\\Kaz Software\end{tabular} & & \begin{tabular}[c]{@{}c@{}}\\Md. Hannan Hossain\\Principal Software Engineer\\ Kaz Software\end{tabular}
    \end{tabular}
\end{center}

\chapter*{Acknowledgement}

I am excited to share this report about my internship experience at Kaz Software.
I want to express my sincere gratitude to the Institute of Information and Communication Technology, SUST, for providing an opportunity for my internship.
I also want to express my regards to Kaz Software for the support and the opportunity to work with a great team.\\

I'd like to offer my heartfelt appreciation to \textbf{Prof M. Jahirul Islam}, Director, Institute of Information and Communication Technology, SUST as well as to my honorable teachers.\\

I would like to thank \textbf{Shawal Siddique Shaon}, CTO of Kaz Software, for his guidance and support throughout the internship period.
I would also like to thank my mentors, \textbf{Md. Hannan Hossain} (Principal Software Engineer) and \textbf{Biswajit Panday} (Senior Software Engineer), who trained and guided me in my internship journey.\\

I want to take this opportunity to thank \textbf{Ibrahim Khan Arshad} (currently working as a software engineer at Cefalo Bangladesh Ltd.) for his kind assistance and support.
Finally, I want to express my gratitude to my team members, my fellow interns, and every member of Kaz Software for making my internship experience a memorable one.
\chapter*{Executive Summary}

The goal of this report is to provide a summary of the work I completed and the experiences I had while working as an intern at Kaz Software, one of Bangladesh's most experienced software company.
This report will go over the specifics of my six-month software engineering internship at Kaz Software, which lasted from September 1st, 2021, to February 28th, 2022.\\


Prior to graduating, students who are studying software engineering at IICT, SUST, must complete a six-month internship program.
As long as the company is reputable, students are free to choose where they will complete their internships, although in most cases, teachers at IIC, SUST will submit their students' resumes to potential employers, who will then select interns from among them.
This is to ensure that the students will have the opportunity to work in an industrial setting.
The purpose of an internship is to expose students to the realities of the workplace, help them comprehend their course material more thoroughly, and offer them the necessary work-related training for the field of software engineering.

\tableofcontents

% Remove if you do not want the table of figures
% \listoffigures

\chapter{Introduction}

\lipsum[1]

\section{Objective}

\lipsum[2]

\subsection{Scope}

\lipsum[3]

\subsubsection{Limitations}

\lipsum[4]

\subsubsection{Transition}

Hello world

\lipsum[2]

\section{Company Profile}

\lipsum[4]

\subsection{Company Type}

\lipsum[4]

\appendix

\chapter{Appendix A}

\lipsum[5]

\section*{Just Testing}

\lipsum[6]


% \bibliographystyle{plain}
% \bibliography{chapters/Biblio.bib}

\end{document}
% -----------------------------------------------------------------
